\documentclass{article}

\usepackage{amsmath}

\begin{document}

\begin{Introduction}
This is a basic introduction to probability theory for an audience which is assumed to be not familiar with mathematical notation. After an intuitive introduction through examples, the topics that I intend to cover are:
\begin{itemize}
	\item Relation between events and sets. 
	\item Introduction of probability as a degree of belief that some event happen. In order to avoid premature abstractions, we restrict to the case of a finite number of events. Many interesting examples fall in this category, even thought the important example of a uniform random variable generated in R is not included. 
	\item Conditional probability, total probability formula and examples. 
	\item Independent events
	\item I want to discuss in detail the case two distinct probability spaces: that is, the probability space whose sample space is the cartesian product of the two spaces. Here we see that t
	\item Bernoulli trials 
	\item Uniform random variable generated in R      
\end{itemize}
I will open a git repository where I will leave the beforehand recorded lectures and where people are invited to leave written questions to which I will try to answer. 





\end{document}
